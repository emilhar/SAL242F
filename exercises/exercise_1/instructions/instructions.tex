\documentclass[11pt]{article}

\usepackage[a4paper,margin=2.5cm]{geometry}
\usepackage[T1]{fontenc}
\usepackage[utf8]{inputenc}
\usepackage{lmodern}
\usepackage{hyperref}
\usepackage{enumitem}
\usepackage{verbatim}

\title{Exercise 1: Introduction to Python for Data Analysis}
\author{}
\date{}

\begin{document}
\maketitle

\section*{Overview}
This exercise is completed in the accompanying Jupyter notebook file called \texttt{sal242f\_exercise1.ipynb}. The exercise is about learning practical Python data analysis with a real dataset:
loading data, inspecting columns, basic cleaning/feature construction, simple research questions, and minimal plotting with \texttt{matplotlib}.

A file ending in \texttt{.ipynb} is a \textbf{Jupyter Notebook}.  
It is an interactive document that mixes text explanations, Python code cells and the output of running those cells. You open it in a notebook environment (e.g.\ Google Colab, JupyterLab, or VS Code) where you can run code cell-by-cell. You can run the notebook wherever you like, but \textbf{Google Colab is recommended} (easy, runs in your browser, no installation).

\subsubsection*{Running a notebook in Google Colab}
\begin{enumerate}[leftmargin=*]
  \item Go to \url{https://colab.research.google.com/}.
  \item Click \textbf{File} $\rightarrow$ \textbf{Upload notebook}.
  \item Upload \texttt{sal242f\_exercise1.ipynb}.
  \item Run a cell with \textbf{Shift+Enter}.
  \item Work from top to bottom. If you change code, run the cell again to update the output.
\end{enumerate}

\subsubsection*{Data files}
The notebook expects a CSV file called \texttt{openalex\_works.csv} to be available locally. You can download the file from: \url{https://www.dropbox.com/scl/fo/puhtq4es2j9rdztw3a8ic/AAKSGYcpYoQm_qsAJGLzyGU?}

Feel free to download the data directly from OpenAlex if you prefer. Visit \url{https://openalex.org/} for more information. Inside the notebook you will see a line like:
\begin{verbatim}
fp = '../../data/exercise_1/openalex_works.csv'  # Change this to the actual path
\end{verbatim}

You must make sure \texttt{fp} points to the correct location of \texttt{openalex\_works.csv} on your system. If you are using Google Colab,you need to upload the CSV file to the Colab environment first.

\vspace{0.5cm}
\textbf{Note:} If you get stuck, you are encouraged to actively seek help:
\begin{itemize}[leftmargin=*]
  \item Look at earlier examples in the provided code.
  \item Search online for information about Python, pandas, Google Colab, etc.
  \item Ask an LLM (for example ChatGPT, Gemini, Copilot, etc.). This is allowed and encouraged for learning. \textbf{Tip when using an LLM:} Paste the question, your code, and the full error message/output. Ask for a hint or a minimal correction and a short explanation.
\end{itemize}

\end{document}
