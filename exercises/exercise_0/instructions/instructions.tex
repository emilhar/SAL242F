\documentclass[11pt]{article}

\usepackage[a4paper,margin=2.5cm]{geometry}
\usepackage[T1]{fontenc}
\usepackage[utf8]{inputenc}
\usepackage{lmodern}
\usepackage{hyperref}
\usepackage{enumitem}

\title{SÁL242F Exercise 0: Python Basics\\\large Instructions}
\author{}
\date{}

\begin{document}
\maketitle

\section*{What you are given}
You are given one file:
\begin{itemize}[leftmargin=*]
  \item A \textbf{Jupyter notebook file} with the extension \texttt{.ipynb}.
\end{itemize}

\section*{What is an \texttt{.ipynb} file?}
An \texttt{.ipynb} file is a \textbf{Jupyter Notebook}.  
It is an interactive document that mixes:
\begin{itemize}[leftmargin=*]
  \item explanatory text (Markdown),
  \item Python code,
  \item and the output of that code.
\end{itemize}

You \textbf{cannot} open an \texttt{.ipynb} file correctly by double-clicking it like a PDF or Word file.  
It must be opened in a notebook environment (such as Google Colab, Jupyter, or VS Code).

\section*{What the notebook contains}
The notebook is a guided, step-by-step introduction to Python basics.  
It is structured as follows:
\begin{itemize}[leftmargin=*]
  \item Short explanations of Python concepts.
  \item \textbf{Task} sections where you write Python code.
  \item Code cells that you run to see results immediately.
\end{itemize}

In the code cells, you will see comments like:
\begin{verbatim}
# TODO: Write your solution here
\end{verbatim}
This is where you write your own code.

\section*{How to view and run the notebook}
You may run the notebook anywhere you like.  
The easiest recommended option is \textbf{Google Colab}, which runs in your browser and requires no installation.

\subsection*{Recommended: Google Colab}
\begin{enumerate}[leftmargin=*]
  \item Go to \url{https://colab.research.google.com/}.
  \item Click \textbf{File} $\rightarrow$ \textbf{Upload notebook}.
  \item Upload the \texttt{.ipynb} file.
  \item The notebook will open in your browser.
  \item Click a cell and press \textbf{Shift+Enter} to run it.
\end{enumerate}


\section*{How to work through the notebook}
\begin{itemize}[leftmargin=*]
  \item Read the explanation text in each section.
  \item For each \textbf{Task}, write code in the cell marked with \texttt{TODO}.
  \item Run the cell and check the output.
  \item Fix errors and re-run as needed.
\end{itemize}


\section*{Getting help (encouraged)}
If you get stuck, you are encouraged to actively seek help:
\begin{itemize}[leftmargin=*]
  \item Re-read the example code in the notebook.
  \item Look up Python basics online.
  \item Ask an LLM (for example ChatGPT, Gemini, Copilot, etc.). This is allowed and encouraged for learning.
\end{itemize}

\noindent\textbf{Tip:} When asking an LLM, include the task description, your current code, and any error messages. Ask for a n explanation, not just the final answer.


\end{document}
