\documentclass[11pt]{article}

\usepackage[a4paper,margin=2.5cm]{geometry}
\usepackage[T1]{fontenc}
\usepackage[utf8]{inputenc}
\usepackage{lmodern}
\usepackage{hyperref}
\usepackage{enumitem}
\usepackage{verbatim}
\usepackage{framed}
\usepackage{tabularx}
\usepackage{url}

% Add spacing between paragraphs
\setlength{\parskip}{0.5em}
%\setlength{\parindent}{0pt}

\title{SÁL242F Exercise 3: Replicating Published Research}
\author{}
\date{}

\begin{document}
\maketitle

\section*{Overview}
In this exercise, are tasked with replicating a published research study. The goal is to get hands-on experience replicating research, identify potential issues in the original study, and learn about the importance of transparency and reproducibility in scientific research.

\section*{Instructions}

Find a published research paper that uses openly available data. You may also choose one of the papers listed below:
\begin{enumerate}[label=\arabic*.]
    \item Spitzer and Mueller (2023). Registered report: Survey on attitudes and experiences regarding preregistration in psychological research. \textit{PLOS ONE}, 18(3), e0281086. \url{https://doi.org/10.1371/journal.pone.0281086}
    \item Trzebiński and Marciniak (2022). There is no smoke without fire: How frequency information and the experience attribution make negative online restaurant reviews more harmful. \textit{PLOS ONE}, 17(7), e0271357. \url{https://doi.org/10.1371/journal.pone.0271357}
    \item Munsch et al. (2019). To eat or not to eat: Reward delay impulsivity in children with loss of control eating, attention deficit / hyperactivity disorder, a double diagnosis, and healthy children. \textit{PLOS ONE}, 14(9), e0221814. \url{https://doi.org/10.1371/journal.pone.0221814}
    \item Klein et al. (2014). Investigating variation in replicability: A “many labs” replication project. \textit{Social Psychology}, 45(3), 142-152. \url{https://doi.org/10.1027/1864-9335/a000178}
\end{enumerate}

\noindent
Your task is to:
\begin{itemize}
    \item Follow the instructions provided in the paper to access the openly available data. Verify that the data is complete and matches the description in the paper.
    \item Rerun the analyses presented in the selected paper using the openly available data.
    \item Compare your results with those reported in the paper.
    \item If you are unable to replicate the results, investigate potential reasons for the discrepancies.
\end{itemize}

\noindent
Prepare a short presentation of your findings, addressing the following questions:
\begin{itemize}
    \item What were the main findings of the original paper?
    \item How did your results compare to those of the original study?
    \item Were you able to replicate the results? If not, what do you think went wrong?
    \item What did you learn from this exercise about the research process and the importance of replication?
\end{itemize}

You are free to use any statistical software or programming language you are comfortable with for the analyses. The use of LLMs for replicating the analyses is encouraged, but make sure you understand every step of the process and can explain it during your presentation.

\noindent\textbf{Good luck!}

\end{document}
